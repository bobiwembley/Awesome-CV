%-------------------------------------------------------------------------------
%	SECTION TITLE
%-------------------------------------------------------------------------------
\cvsection{Expérience Professionnelle}


%-------------------------------------------------------------------------------
%	CONTENT
%-------------------------------------------------------------------------------
\begin{cventries}

%---------------------------------------------------------
  \cventry
    {Consultant Indépendant - Devops  Ingénieur Automatisation - }  % Job title
    {Pour la division BNP PARIBAS CIB} % Organization
    {Paris, France} % Location
    {Déc. 2023 - Déc. 2024} % Date(s)
    {
      \begin{cvitems} % Description(s) of tasks/responsibilities
        \item {Implémentation et maintenance des solutions Dynatrace pour une surveillance complète des applications et de l'infrastructure.}
        \item {Utilisation de Git pour le contrôle de version, en maintenant les fichiers de configuration et les scripts liés aux configurations Dynatrace.}
        \item {Utilisation d'Ansible pour l'automatisation, rationalisant le déploiement et la gestion de Dynatrace sur divers environnements.}
        \item {Collaboration avec les membres de l'équipe pour assurer une surveillance efficace des performances et une observabilité dans toute l'organisation.}
        \item {optimisation des performances système et l'identification des domaines à améliorer grâce aux analyses Dynatrace.}
        \item {Travail collaboratif pour améliorer les stratégies d'observabilité, assurant une approche proactive de la détection et de la résolution des problèmes.}
        \item {Contribué au développement et à la mise en œuvre des meilleures pratiques pour l'utilisation des outils Dynatrace au sein de l'équipe.}
      \end{cvitems}
    }

%---------------------------------------------------------
  \cventry
    {Ingénieur Fiabilité des Sites (SRE)} % Job title
    {Adyoulike Inc.} % Organization
    {Paris, France} % Location
    {Fév. 2023 - Août 2023} % Date(s)
    {
      \begin{cvitems} % Description(s) of tasks/responsibilities
        \item {En tant qu'Ingénieur Fiabilité des Sites, gestion d'un environnement multicloud englobant OVH, AWS, GCP, et une infrastructure sur site.}
        \item {Maintenance et optimisation des performances et de la fiabilité de technologies critiques, dont Apache Mesos, Kafka, Docker, et Hadoop pour le traitement de données volumineuses.}
        \item {Implémentation réussie de ScyllaDB en lieu et place de  MongoDB sur Ubuntu 18.04, améliorant l'efficacité et la scalabilité de la base de données.}
        \item {Conduite de la migration d'Apache Airflow vers Google Kubernetes Engine (GKE) sur Google Cloud Platform (GCP), assurant une transition transparente et des capacités d'orchestration améliorées.}
        \item {Collaboration avec des équipes pluridisciplinaires pour maintenir et soutenir les systèmes existants, garantissant une disponibilité élevée et une fiabilité dans une architecture multicloud complexe.}
        \item {parti à la  conception et l'exécution d'une stratégie de migration, traitant des défis spécifiques à un environnement multicloud et une pile technologique diversifiée.}
        \item {Contribué au développement des meilleures pratiques SRE, mettant l'accent sur la fiabilité, la scalabilité et une réponse efficace aux incidents.}
      \end{cvitems}
      
    }

%---------------------------------------------------------
  \cventry
    {Ingénieur d'Intégration \& Administrateur Big Data} % Job title
    {Société Générale} % Organization
    {Paris, France} % Location
    {Mars 2020 - Sept. 2022} % Date(s)
    {
      \begin{cvitems} % Description(s) of tasks/responsibilities
        \item {En tant qu'Ingénieur d'Intégration dans un environnement de production Hadoop, j'ai joué un rôle crucial dans la conception, le déploiement et l'optimisation de systèmes complexes de traitement de données.}
        \item {Utilisé Ansible et Git pour automatiser et accélérer le processus de livraison, en particulier dans un domaine commercial hautement critique axé sur la détection de la fraude bancaire.}
        \item {Maintien de Kafka comme pilote d'événements pour le streaming de données en temps réel, permettant une communication et un flux de données efficaces au sein de l'écosystème Hadoop.}
        \item {Implémenté HBase comme une base de données NoSQL, offrant un stockage évolutif et distribué pour de grands volumes de données générées dans le cadre de l'analyse de la fraude bancaire.}
        \item {Participé à la première migration des pratiques de déploiement héritées vers une approche moderne de l'Intégration Continue/Deploiement Continue (CI/CD).}
        \item {Promovoir l'intégration des pratiques CI/CD pour rationaliser et améliorer le cycle de vie du développement logiciel.}
        \item {Fait face à des défis lors du processus d'adoption de la CI/CD, notamment surmonter les dépendances des systèmes hérités, garantir la compatibilité ascendante et s'aligner sur des exigences strictes de sécurité et de conformité dans le domaine de la détection de la fraude bancaire.}
        \item {Collaboré avec des équipes pluridisciplinaires pour résoudre les difficultés, établir des meilleures pratiques et assurer une transition en douceur vers des déploiements CI/CD modernes.}
        \item {Contribué au développement de pipelines CI/CD robustes, intégrant des tests automatisés, des stratégies de déploiement et un contrôle de version pour garantir la fiabilité et l'efficacité des versions logicielles.}
      \end{cvitems}
      
    }

%---------------------------------------------------------

\cventry
{Ingénieur Infrastructure} % Titre du poste
{BNP Paribas et la Division BP2I} % Organisation
{Paris, France} % Location
{Jan. 2018 - Mar. 2019} % Date(s)
{
  \begin{cvitems} % Description(s) des tâches/responsabilités
    \item {Maintien systèmes et services de Splunk sur site pour BNP Paribas et sa division BP2I.}
   \item {Travail sur l'optimisation des performances du système, identification et résolution des problèmes liés à l'infrastructure.}
    \item {Encadrement de la mise en œuvre de Git pour contrôler le code dans une perspective d'intégration continue (CI).}
    \item {Déploiement et tests en CI/CD à l'aide des pipelines GitLab CI.}
    \item {Déploiement de l'infrastructure avec Ansible, assurant des processus de déploiement efficaces et cohérents.}
    \item {Collaboration avec des équipes interfonctionnelles pour concevoir et déployer une infrastructure évolutive et sécurisée.}
    \item {Collaboration avec les membres de l'équipe pour améliorer l'observabilité générale du système et la surveillance.}
  \end{cvitems}
}

%---------------------------------------------------------
\cventry
{Système Administrateur} % Job title
{Synalabs} % Organization
{Paris, France} % Location
{Sep. 17 - Dec. 17 } % Date(s)
{
  \begin{cvitems} % Description(s) of tasks/responsibilities
    \item {Gestion et administration des systèmes d'hébergement web pour divers clients de Synalabs.}
    \item {Configuration, maintenance et surveillance des serveurs web, notamment Apache et Nginx.}
    \item {Installation et gestion de bases de données, notamment MySQL et PostgreSQL, pour assurer des performances optimales des sites web hébergés.}
    \item {Mise en place de mesures de sécurité robustes, y compris la configuration des pare-feu, la gestion des certificats SSL, et la surveillance proactive des vulnérabilités.}
    \item {Automatisation des tâches récurrentes à l'aide de scripts personnalisés et d'outils d'automatisation.}
    \item {Assistance technique aux clients pour résoudre les problèmes liés à l'hébergement, y compris la résolution des problèmes de connectivité, de performance et de sécurité.}
    \item {Collaboration avec les équipes de développement pour assurer le déploiement et la gestion efficaces des applications web.}
    \item {Mise en place de stratégies de sauvegarde fiables et récupération des données en cas de sinistre.}
  \end{cvitems}
}

%---------------------------------------------------------
\cventry
{Consultant Indépendant - DevOps Engineer -} % Job title
{Deveryware} % Organization
{Paris, France} % Location
{Juin 16 - Aout 17} % Date(s)
{
  \begin{cvitems} % Description(s) of tasks/responsibilities
    \item {Implémentation d'un système d'authentification double pour renforcer la sécurité d'une application.}
    \item {Maintenance proactive de la solution de géolocalisation pour assurer disponibilité et fiabilité.}
    \item {Utilisation d'un environnement entièrement open source : Proxmox, GitLab CE, Jenkins.}
    \item {Déploiement et gestion de l'infrastructure avec des outils HashiCorp pour garantir la scalabilité et la sécurité.}
    \item {Travail étroit avec l'équipe pour établir les meilleures pratiques de DevOps dans les flux de travail du projet.}
    \item {Intégration et optimisation des outils HashiCorp pour améliorer les processus de déploiement et de gestion.}
    \item {Contributions à l'amélioration continue des pratiques de l'équipe et à l'efficacité globale du projet.}
  \end{cvitems}
}

%---------------------------------------------------------
\cventry
{Consultant Indépendant - Ingénieur Système - } % Titre du poste
{AB Tasty} % Organisation
{Paris, France} % Lieu
{Dec. 15 - Juin 16} % Dates
{
  \begin{cvitems} % Description(s) des tâches/responsabilités
    \item {Gestion et maintenance des bases de données MySQL dans un environnement multi-cloud.}
    \item {Concentration principalement sur les services hébergés sur AWS, en utilisant Elastic Beanstalk pour le déploiement.}
    \item {Travail approfondi avec la base de données Aurora pour le stockage et la récupération des données.}
    \item {Introduction de Docker dans les pratiques d'infrastructure pour améliorer la scalabilité et l'efficacité.}
    \item {Conduire le processus de migration de MySQL vers MariaDB pour des solutions de base de données améliorées.}
    \item {Collaboration avec des équipes interfonctionnelles pour assurer une intégration transparente des services sur AWS.}
    \item {Contributions à l'adoption de Docker pour la conteneurisation au sein de l'infrastructure.}
  \end{cvitems}
}
%---------------------------------------------------------


\end{cventries}


