%-------------------------------------------------------------------------------
%	TITRE DE LA SECTION
%-------------------------------------------------------------------------------
\cvsection{Expérience Professionnelle}

%-------------------------------------------------------------------------------
%	CONTENU
%-------------------------------------------------------------------------------
\begin{cventries}

%---------------------------------------------------------

%---------------------------------------------------------
\cventry
{Consultant Freelance DevOps \& Ingénieur en Automatisation} % Intitulé du poste
{BNP PARIBAS, division CIB} % Organisation
{Paris, France} % Lieu
{Déc. 2023 - Déc. 2024} % Dates
{\textbf{Role:} membre de l"équipe Observability \& Monitoring.}
{
  \begin{cvitems}
    \item {Dans le cadre de la migration vers La solution Dynatrace pour le monitoring }
    \item {Collaboration avec les membres de l'équipe pour assurer une surveillance efficace des performances et une observabilité dans toute l'organisation.}
    \item {Mise en place et maintenance des solutions Dynatrace pour une surveillance complète des applications et de l'infrastructure.}
    \item {Utilisation de Git pour le contrôle de version, en maintenant les fichiers de configuration et les scripts liés aux configurations Dynatrace.}
    \item {Utilisation d'Ansible pour l'automatisation, rationalisant le déploiement et la gestion de Dynatrace sur différents environnements.}
    \item {Joué un rôle clé dans l'optimisation des performances du système et l'identification des domaines à améliorer grâce aux informations de Dynatrace.}
    \item {Travail collaboratif pour améliorer les stratégies d'observabilité globales, assurant une approche proactive pour la détection et la résolution des problèmes.}
    \item {Contribué au développement et à la mise en œuvre des meilleures pratiques pour l'utilisation des outils Dynatrace au sein de l'équipe.}
  \end{cvitems}
}

%---------------------------------------------------------
\cventry
{Consultant Freelance DevOps} % Intitulé du poste
{Canal Plus. (Groupe Bolloré)} % Organisation
{Paris, France} % Lieu
{Août 2023 - Déc. 2023} % Dates
{
  \begin{cvitems} % Description(s) des tâches/responsabilités
    \item {En tant qu'ingénieur DevOps, j'ai dirigé la mise en place de pipelines CI/CD de bout en bout en utilisant des outils tels que Ansible, Jenkins, Git et Jira.}
    \item {Utilisation de Git pour le contrôle de version, mise en œuvre et maintenance de flux de travail CI/CD pour le déploiement efficace du code de la solution.}
    \item {Introduction et promotion des meilleures pratiques en matière de CI/CD, en prêchant l'importance de l'automatisation et de l'intégration continue tout au long du cycle de développement.}
    \item {Rôle clé dans la migration des systèmes de contrôle de version, menant la transition de Bitbucket à GitLab. Réalisation réussie d'une démonstration de faisabilité pour la migration des processus CI/CD de Jenkins à GitLab CI.}
    \item {Introduction de SonarQube pour l'analyse de la qualité du code, améliorant l'ensemble du processus de développement.}
    \item {Collaboration avec des équipes interfonctionnelles, en utilisant Jira pour une gestion de projet efficace et le suivi des problèmes.}
    \item {Optimisation continue et rationalisation des processus de déploiement, assurant une livraison plus rapide et plus fiable des solutions logicielles.}
    \item {Contribution au développement et à la mise en œuvre des meilleures pratiques pour les processus DevOps au sein de l'équipe et de l'organisation.}
    \item {Travail étroit avec différents services pour assurer une transition fluide vers un modèle de cloud hybride, répondant aux exigences commerciales uniques et favorisant la collaboration entre les équipes.}
  \end{cvitems}        
}

%---------------------------------------------------------
\cventry
{Ingénieur Fiabilité des Sites (SRE)} % Intitulé du poste
{Adyoulike Inc.} % Organisation
{Paris, France} % Lieu
{Fév. 2023 - Août 2023} % Dates
{
  \begin{cvitems} % Description(s) des tâches/responsabilités
    \item {En tant qu'Ingénieur Fiabilité des Sites, gestion d'un environnement multi-cloud couvrant OVH, AWS, GCP, et une infrastructure sur site.}
    \item {Maintenance et optimisation des performances et de la fiabilité des technologies critiques, dont Apache Mesos, Kafka, Docker, et Hadoop pour le traitement des données volumineuses.}
    \item {Implémentation réussie de ScyllaDB sur MongoDB sur Ubuntu 18.04, améliorant l'efficacité et la scalabilité de la base de données.}
    \item {Direction de la migration d'Apache Airflow vers Google Kubernetes Engine (GKE) sur la plateforme Google Cloud (GCP), assurant une transition sans heurts et des capacités d'orchestration améliorées.}
    \item {Collaboration avec des équipes interfonctionnelles pour maintenir et prendre en charge les systèmes existants, assurant une disponibilité et une fiabilité élevées dans une architecture multi-cloud complexe.}
    \item {Rôle clé dans la création et l'exécution d'une stratégie de migration, abordant les défis spécifiques d'un environnement multi-cloud et d'une pile technologique diversifiée.}
    \item {Contribution au développement des meilleures pratiques SRE, mettant l'accent sur la fiabilité, la scalabilité et une réponse aux incidents efficace.}
  \end{cvitems}
}

%---------------------------------------------------------
\cventry
{Ingénieur d'Intégration et Administrateur Big Data} % Intitulé du poste
{Société Générale} % Organisation
{Paris, France} % Lieu
{Mar. 2020 - Sept. 2022 } % Dates
{
  \begin{cvitems} % Description(s) des tâches/responsabilités
    \item {En tant qu'ingénieur d'intégration dans un environnement de production Hadoop, j'ai joué un rôle crucial dans la conception, le déploiement et l'optimisation de systèmes complexes de traitement des données.}
    \item {Utilisation d'Ansible et de Git pour automatiser et accélérer le processus de livraison, en particulier dans un domaine métier à haut risque axé sur la détection de la fraude bancaire.}
    \item {Maintenance de Kafka en tant que moteur d'événements pour le streaming de données en temps réel, permettant une communication et un flux de données efficaces au sein de l'écosystème Hadoop.}
    \item {Maintenance de HBase en tant que base de données NoSQL, fournissant un stockage évolutif et distribué pour de grands volumes de données générés dans le contexte de l'analyse de la fraude bancaire.}
    \item {Leadership dans la première migration des pratiques de déploiement héritées vers une approche moderne de l'Intégration Continue/Intégration Continue (CI/CD).}
    \item {Promotion de l'intégration des pratiques CI/CD pour rationaliser et améliorer le cycle de développement logiciel.}
    \item {Confrontation à des défis lors du processus d'adoption CI/CD, notamment la résolution de dépendances des systèmes hérités, garantissant la compatibilité ascendante et l'alignement sur des exigences de sécurité et de conformité strictes dans le domaine de la détection de la fraude bancaire.}
    \item {Collaboration avec des équipes interfonctionnelles pour résoudre des difficultés, établir des meilleures pratiques et assurer une transition en douceur vers des builds CI/CD modernes.}
    \item {Contribution au développement de pipelines CI/CD robustes, intégrant des tests automatisés, des stratégies de déploiement et un contrôle de version pour garantir la fiabilité et l'efficacité des versions logicielles.}
  \end{cvitems}  
}

%---------------------------------------------------------
\cventry
{Ingénieur Infrastructure} % Intitulé du poste
{BNP Paribas et la division BP2I} % Organisation
{Paris, France} % Lieu
{Jan. 2018 - Mar. 2019} % Dates
{
  \begin{cvitems} % Description(s) des tâches/responsabilités
    \item {Mise en place de solutions Splunk sur site pour BNP Paribas et sa division BP2I.}
    \item {Direction de la mise en œuvre de Git pour contrôler le code dans une perspective CI.}
    \item {Déploiement et tests en CI/CD à l'aide de pipelines GitLab CI.}
    \item {Déploiement de l'infrastructure à l'aide d'Ansible, assurant des processus de déploiement efficaces et cohérents.}
    \item {Collaboration avec des équipes interfonctionnelles pour concevoir et déployer une infrastructure évolutive et sécurisée.}
    \item {Utilisation de systèmes de contrôle de version, en particulier Git, pour gérer les fichiers de configuration et le code d'infrastructure.}
    \item {Travail sur l'optimisation des performances du système, l'identification et la résolution des problèmes liés à l'infrastructure.}
    \item {Contribution au développement et à la mise en œuvre des meilleures pratiques pour la gestion de l'infrastructure.}
    \item {Collaboration avec les membres de l'équipe pour améliorer l'observabilité générale du système et la surveillance.}
  \end{cvitems}
}

%---------------------------------------------------------
\cventry
{Administrateur Système} % Intitulé du poste
{Synalabs} % Organisation
{Ville, Pays} % Lieu
{Date de début - Date de fin} % Dates
{
  \begin{cvitems} % Description(s) des tâches/responsabilités
    \item {Gestion et administration des systèmes d'hébergement web pour divers clients chez Synalabs.}
    \item {Configuration, maintenance et surveillance des serveurs web, y compris Apache et Nginx.}
    \item {Installation et gestion de bases de données, y compris MySQL et PostgreSQL, pour assurer des performances optimales des sites web hébergés.}
    \item {Mise en place de mesures de sécurité robustes, y compris la configuration du pare-feu, la gestion des certificats SSL et la surveillance proactive des vulnérabilités.}
    \item {Automatisation des tâches récurrentes à l'aide de scripts personnalisés et d'outils d'automatisation.}
    \item {Fourniture d'un support technique aux clients pour résoudre les problèmes liés à l'hébergement, y compris la connectivité, les performances et la résolution des problèmes de sécurité.}
    \item {Collaboration avec les équipes de développement pour assurer le déploiement et la gestion efficaces des applications web.}
    \item {Mise en place de stratégies fiables de sauvegarde et de récupération des données en cas de catastrophe.}
  \end{cvitems}
}

%---------------------------------------------------------
\cventry
{Ingénieur DevOps} % Intitulé du poste
{Deveryware} % Organisation
{Paris, France} % Lieu
{Juin 2016 - Août 2017} % Dates
{
  \begin{cvitems} % Description(s) des tâches/responsabilités
    \item {Mise en place d'un système d'authentification double pour améliorer la sécurité d'une application.}
    \item {Maintenance proactive de la solution de géolocalisation, garantissant la disponibilité et la fiabilité.}
    \item {Utilisation d'un environnement entièrement open-source : Proxmox, GitLab CE, Jenkins.}
    \item {Déploiement et gestion de l'infrastructure à l'aide des outils HashiCorp pour assurer la scalabilité et la sécurité.}
    \item {Travail étroit avec l'équipe pour établir les meilleures pratiques DevOps dans les flux de travail du projet.}
    \item {Intégration et optimisation des outils HashiCorp pour améliorer les processus de déploiement et de gestion.}
    \item {Contribution à l'amélioration continue des pratiques de l'équipe et de l'efficacité globale du projet.}
  \end{cvitems}
}

%---------------------------------------------------------
\cventry
{Consultant Système} % Intitulé du poste
{AB Tasty} % Organisation
{Paris, France} % Lieu
{Déc. 2015 - Juin 2016} % Dates
{
  \begin{cvitems} % Description(s) des tâches/responsabilités
    \item {Gestion et maintenance des bases de données MySQL dans un environnement multi-cloud.}
    \item {Concentration principale sur les services hébergés sur AWS, utilisant Elastic Beanstalk pour le déploiement.}
    \item {Travail approfondi avec la base de données Aurora pour le stockage et la récupération des données.}
    \item {Introduction de Docker dans les pratiques d'infrastructure pour améliorer la scalabilité et l'efficacité.}
    \item {Direction du processus de migration de MySQL vers MariaDB pour des solutions de base de données améliorées.}
    \item {Collaboration avec des équipes interfonctionnelles pour assurer l'intégration transparente des services sur AWS.}
    \item {Contribution à l'adoption de Docker pour la conteneurisation au sein de l'infrastructure.}
  \end{cvitems}
}

%---------------------------------------------------------
\cventry
{Ingénieur d'Exploitation} % Intitulé du poste
{Meetic} % Organisation
{Paris, France} % Lieu
{Mars 2015 - Déc. 2015} % Dates
{
  \begin{cvitems} % Description(s) des tâches/responsabilités
    \item {Gestion de l'infrastructure sur site pour Meetic, garantissant des performances optimales et une fiabilité continue.}
    \item {Maintenance et support de l'infrastructure de solution de Meetic avec une surveillance continue et une résolution proactive des problèmes.}
    \item {Participation active au sein de l'équipe pour fournir des solutions multi-environnementales, adaptant l'infrastructure aux besoins opérationnels divers.}
    \item {Collaboration étroite avec des équipes interfonctionnelles pour répondre et satisfaire aux exigences de l'infrastructure.}
    \item {Utilisation de meilleures pratiques pour assurer la scalabilité et la sécurité de l'infrastructure soutenant les services de Meetic.}
    \item {Contribution à l'amélioration continue des processus d'infrastructure et de l'efficacité opérationnelle.}
  \end{cvitems}
}

%-----------------------------------------------------------------
\end{cventries}
